%
% @author   Christopher Schmitt & Matthew Warren
% @version  3/29/2020
% @license  MIT
%


\documentclass{article}


%
% Document Imports
%

\usepackage{fancyhdr}
\usepackage{extramarks}
\usepackage{amsmath}
\usepackage{amssymb}
\usepackage{amsthm}
\usepackage{amsfonts}
\usepackage{color}
\usepackage{listings}
\usepackage[color]{register}
\usepackage{booktabs}



%
% Document Configuration
%

\newcommand{\hwAuthor}{Christopher K. Schmitt and Matthew Warren}
\newcommand{\hwSubject}{CS 358}
\newcommand{\hwSection}{Section 01}
\newcommand{\hwSemester}{Spring 2020}
\newcommand{\hwAssignment}{Progress Report 2}

\definecolor{dkgreen}{rgb}{0,0.6,0}
\definecolor{gray}{rgb}{0.5,0.5,0.5}
\definecolor{mauve}{rgb}{0.58,0,0.82}

\lstset{
  frame=tb,
  language=Verilog,
  aboveskip=3mm,
  belowskip=3mm,
  showstringspaces=false,
  columns=flexible,
  basicstyle=\ttfamily,
  numbers=none,
  numberstyle=\tiny\color{gray},
  keywordstyle=\color{blue},
  commentstyle=\color{dkgreen},
  stringstyle=\color{mauve},
  breaklines=true,
  breakatwhitespace=true,
  tabsize=3
}


%
% Document Environments
%

\setlength{\headheight}{65pt}
\pagestyle{fancy}
\lhead{\hwAuthor}
\rhead{
  \hwSubject \\
  \hwSection \\
  \hwSemester \\
  \hwAssignment
}

\newenvironment{problem}[1]{
  \nobreak\section*{#1}
}{}


%
% Document Start
%

\begin{document}
  \begin{problem}{Instruction Set Architecture}
    \paragraph{}
    We implement a 3-satge MIPS data-path which implements all R-Type instructions and ADDI.

    \begin{table}[]
      \begin{tabular}{@{}lllp{6.8cm}@{}}
        \toprule
        Instruction & Opcode & Format & Description \\ \midrule
        add & 0000 & R-Type & Adds \$rs and \$rt together using arithmetic addition, places the sum in \$rt \\
        sub & 0001 & R-Type & Subtract \$rt from \$rs, places the difference in \$rt \\
        and & 0010 & R-Type & Perform bitwise ANDing on \$rs and \$rt, place the result in \$rd \\
        or & 0011 & R-Type & Perform bitwise ORing on \$rs and \$rt, place the result in \$rd \\
        addi & 0100 & I-Type & Add \$rs to immediate value, place the result in \$rt \\
        slt & 0111 & R-Type & Put 0x01 in \$rd if \$rs $<$ \$rt, 0x00 otherwise \\ \bottomrule
      \end{tabular}
    \end{table}
    
    \begin{register}{H}{R-Type}{}
      \begin{center}
        \regfield[green!30]{}{4}{12}{{op}}
        \regfield[red!30]{}{2}{10}{{rs}}
        \regfield[blue!30]{}{2}{8}{{rt}}
        \regfield[orange!30]{}{2}{6}{{rd}}
        \regfield[gray!30]{}{6}{0}{{unused}}
      \end{center}
    \end{register}

    \begin{register}{H}{I-Type}{}
      \begin{center}
        \regfield[green!30]{}{4}{12}{{op}}
        \regfield[red!30]{}{2}{10}{{rs}}
        \regfield[blue!30]{}{2}{8}{{rt}}
        \regfield[orange!30]{}{8}{0}{{value}}
      \end{center}
    \end{register}

    In the above diagrams, \emph{op} is the opcode, \emph{rs} is the
    source register, \emph{rt} is the target/destination register, and
    \emph{rd} is the destination register.  Value is the immediate value
    in I-Type instructions.
  \end{problem}

  \begin{problem}{Control Table}
    \begin{table}[]
      \begin{tabular}{@{}llllllll@{}}
      Operation & RegDst & ALUSrc & MemToReg & RegWrite & MemWrite & Branch & ALUOp \\ \midrule
      add & 1 & 0 & 0 & 1 & 0 & 0 & 00 \\
      sub & 1 & 0 & 0 & 1 & 0 & 0 & 01 \\
      and & 1 & 0 & 0 & 1 & 0 & 0 & 10 \\
      or & 1 & 0 & 0 & 1 & 0 & 0 & 10 \\
      addi & 0 & 1 & 0 & 1 & 0 & 0 & 00 \\
      lw & 0 & 1 & 1 & 1 & 0 & 0 & 00 \\
      sw & 0 & 0 & 0 & 0 & 1 & 0 & 00 \\
      slt & 1 & 0 & 0 & 1 & 0 & 0 & 10 \\
      beq & 0 & 0 & 0 & 0 & 0 & 1 & 01 \\
      bne & 0 & 0 & 0 & 0 & 0 & 1 & 01
      \end{tabular}
    \end{table}
  \end{problem}

  \begin{problem}{Source Code}
    \lstinputlisting[language=Verilog, basicstyle=\scriptsize\ttfamily]{processor.vl}
  \end{problem}

  \begin{problem}{Machine Translation}
    \lstinputlisting[basicstyle=\ttfamily]{test.txt}
  \end{problem}

  \begin{problem}{Output (Branch Taken)}
    \begin{center}
      \begin{lstlisting}[basicstyle=\footnotesize\ttfamily]
time pc  IF_ID ID_EX EX_MEM MEM_WB writeData
 0    0  0000  0000  xxxx   xxxx   xxxx
 1    1  5100  0000  0000   xxxx   xxxx
 3    2  5201  5100  0000   0000   xxxx
 5    3  0000  5201  5100   0000   0000
 7    4  0000  0000  5201   5100   0007
 9    5  0000  0000  0000   5201   0005
11    6  76c0  0000  0000   0000   0000
13    7  0000  76c0  0000   0000   0000
15    8  0000  0000  76c0   0000   0000
17    9  0000  0000  0000   76c0   0000
19   10  8c05  0000  0000   0000   0000
21   11  0000  8c05  0000   0000   0000
23   12  0000  0000  8c05   0000   0000
25   15  0000  0000  0000   8c05   0000
27   16  0000  0000  0000   0000   0000
29   17  0000  0000  0000   0000   0000
31   18  0000  0000  0000   0000   0000
33   19  5100  0000  0000   0000   0000
35   20  5201  5100  0000   0000   0000
37   21  0000  5201  5100   0000   0000
39   22  0000  0000  5201   5100   0007
41   23  0000  0000  0000   5201   0005
43   24  16c0  0000  0000   0000   0000
45   25  xxxx  16c0  0000   0000   0000
47   26  xxxx  xxxx  16c0   0000   0000
49   27  xxxx  xxxx  xxxx   16c0   0002
51   28  xxxx  xxxx  xxxx   xxxx   xxxx
53   29  xxxx  xxxx  xxxx   xxxx   xxxx
55   30  xxxx  xxxx  xxxx   xxxx   xxxx
      \end{lstlisting}
    \end{center}
  \end{problem}

  \begin{problem}{Output (Branch Not Taken)}
    \begin{center}
      \begin{lstlisting}[basicstyle=\footnotesize\ttfamily]
time pc  IF_ID ID_EX EX_MEM MEM_WB writeData
0     0  0000  0000  xxxx   xxxx   xxxx
1     1  5100  0000  0000   xxxx   xxxx
3     2  5201  5100  0000   0000   xxxx
5     3  0000  5201  5100   0000   0000
7     4  0000  0000  5201   5100   0005
9     5  0000  0000  0000   5201   0007
11    6  76c0  0000  0000   0000   0000
13    7  0000  76c0  0000   0000   0000
15    8  0000  0000  76c0   0000   0000
17    9  0000  0000  0000   76c0   0001
19   10  8c05  0000  0000   0000   0000
21   11  0000  8c05  0000   0000   0000
23   12  0000  0000  8c05   0000   0000
25   13  0000  0000  0000   8c05   0001
27   14  6100  0000  0000   0000   0000
29   15  6201  6100  0000   0000   0000
31   16  0000  6201  6100   0000   0000
33   17  0000  0000  6201   6100   0000
35   18  0000  0000  0000   6201   0001
37   19  5100  0000  0000   0000   0000
39   20  5201  5100  0000   0000   0000
41   21  0000  5201  5100   0000   0000
43   22  0000  0000  5201   5100   0005
45   23  0000  0000  0000   5201   0007
47   24  16c0  0000  0000   0000   0000
49   25  xxxx  16c0  0000   0000   0000
51   26  xxxx  xxxx  16c0   0000   0000
53   27  xxxx  xxxx  xxxx   16c0   fffe
55   28  xxxx  xxxx  xxxx   xxxx   xxxx
      \end{lstlisting}
    \end{center}
  \end{problem}
\end{document}